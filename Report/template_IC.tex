\documentclass[12pt,a4paper]{article}

% Pacotes para o português.
\usepackage[brazilian]{babel}
\usepackage[utf8]{inputenc}
\usepackage[T1]{fontenc}

\usepackage{graphicx}
\usepackage{xcolor}
\usepackage{indentfirst}
\usepackage{url}
\usepackage{array}
\usepackage[top=2cm, bottom=2cm, left=2cm, right=2cm]{geometry}
\usepackage{multirow}
\usepackage{amssymb}
\usepackage{amsmath}
\usepackage{caption}
\usepackage{setspace}
\usepackage{breakcites}
\usepackage{float}
\usepackage{times}
\usepackage{lipsum}

% Comando para marcar o texto para revisão.
\newcommand{\rev}[1]{\textcolor{red}{#1}}

% Permite escrever aspas normais "text" em vez de ``text''
\usepackage[autostyle]{csquotes}
\MakeOuterQuote{"}

\begin{document}

\doublespacing

\begin{titlepage}
    \begin{center}
        {\large \sc UNIVERSIDADE DE SÃO PAULO} \\
        {\large \sc INSTITUTO DE CIÊNCIAS MATEMÁTICAS E DE COMPUTAÇÃO}\\[0.7cm]
        {\small \sc DEPARTAMENTO DE .... }\\[2.8cm]

        % Título.
        {\large \sc Projeto de Pesquisa de Iniciação Científica - Sem Bolsa}\\
        \rule{0.9\linewidth}{0.5mm} \\[0.4cm]
        {\large \bfseries Título}\\
        \rule{0.9\linewidth}{0.5mm} \\[0.4cm]
    \end{center}
    
    % Assinaturas
    \begin{minipage}{0.45\textwidth}
        \emph{Candidato:}\\[2.08cm]
        \begin{picture}(5,20)(0,-22.5) 
            % \put(0,1){\includegraphics[scale=0.13]{figs/assinatura-aluno.jpg}}
            \put(0,0){\rule{0.9\linewidth}{0.5mm}}
            \put(0,-22.5){}
        \end{picture}
    \end{minipage}
    \hspace{1cm}
    \begin{minipage}{0.45\textwidth}
        \emph{Orientador:}\\[2.08cm]
        \begin{picture}(130,20)(0,-22.5)
            % \put(18,1){\includegraphics[scale=0.48]{figs/assinatura-orientador.png}}
            \put(0,0){\rule{0.9\linewidth}{0.5mm}}
            \put(0,-22.5){}
        \end{picture}
    \end{minipage}

    \vfill

    % Data
    \begin{center}
        \makeatletter
        \@date
        \makeatother
    \end{center}
\end{titlepage}

\pagestyle{empty}
\begin{center}
    {\bf \Large Resumo}
\end{center}
%

\noindent{}
\newpage
\pagestyle{empty}
\tableofcontents
\newpage
\setcounter{page}{1}
\pagestyle{plain}

\section{Introdução}
\label{section:introducao}

\section{Trabalhos Relacionados}
\label{section:trabrelacionados}

\section{Objetivos}
\label{section:objetivos}
\subsection{Objetivos gerais}

\subsection{Objetivos específicos}

\section{Plano de Trabalho e Cronograma} 

\section{Materiais e Métodos}
\label{section:materiais}
Não existe uma "receita pronta" para escrever a seção Materiais e Métodos que funcione para todos os trabalhos. Logicamente, quanto mais voltado às aplicações for seu projeto, mais relevante será a explicação e o nível de detalhe desta seção. Ela deveria possuir um nível de detalhamento tal que, se o projeto for entregue a outra pessoa, ela consiga executar a pesquisa exatamente da mesma forma que você executaria (Vianna, 2001). Na parte de métodos podem entrar outros itens, tais como a apresentação de seminários para o orientador, a resolução de problemas, etc. 

\newpage

\begin{thebibliography}{99}
\bibitem{artigo} T. Izio and C. Aio, Novos e Importantes Resultados, \emph{Int. Journal of Mickey Mouse} {\bf 60}  (1988), 185-195.
\bibitem{livro} A.~Utor,  \emph{Assuntos metarelevantes}, Springer-Verlag, Berlin and Heidelberg, 1993. vii+260 pp.
\end{thebibliography}

\end{document}
